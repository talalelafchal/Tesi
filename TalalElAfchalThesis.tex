\documentclass[12pt,mscthesis]{usiinfthesis}

\usepackage{lipsum}


\usepackage{listings}
\usepackage[autostyle]{csquotes} 

\lstdefinelanguage{algebra}
{morekeywords={import,sort,constructors,observers,transformers,axioms,if,
else,end},
sensitive=false,
morecomment=[l]{//s},
}



\title{Assessing Documents by Comprehension Effort } %compulsory
%\specialization{Dependable Distributed Systems}%optional
%\subtitle{Subtitle: Reinventing the World} %optional 
\author{Talal El Afchal} %compulsory
\begin{committee}
\advisor{Prof.}{Michele}{Lanza} %compulsory
\coadvisor{Prof.}{Gabriele}{Bavota}{} %optional
\end{committee}
\Day{1} %compulsory
\Month{September} %compulsory
\Year{2017} %compulsory, put only the year
\place{Lugano} %compulsory

\dedication{To my beloved} %optional
\openepigraph{Your living is determined not so much by what life brings to you as by the attitude you bring to life; not so much by what happens to you as by the way your mind looks at what happens}{Gubran Khalil Gubran} %optional

%\makeindex %optional, also comment out \theindex at the end

\begin{document}

\maketitle %generates the titlepage, this is FIXED

\frontmatter %generates the frontmatter, this is FIXED

\begin{abstract}
Recommender systems for software developers have become increasingly popular in recent years. These systems combine several methodologies to provide suggestions that meet the developer's needs. The recommender systems collect data from online resources as blogs, forums, Q\&A websites, and suggest documents or piece of code that are most likely helpful to the developers. However, these systems are not taking into consideration an important aspect as the comprehension effort which may vary depending on the document familiarity and readability. Usually developers are more interested in documents which they are familiar with. By calculating the comprehension effort, the recommender system can re-rank the documents and suggest the most comprehensive and appropriate ones to the developer. In this work, we present our approach to calculating the effort, by creating a language model able to capture a document familiarity, that we combine with the document readability. 
\end{abstract}

\begin{acknowledgements}
\end{acknowledgements}

\tableofcontents 
\listoffigures %optional
\listoftables %optional

\mainmatter

\chapter{Introduction}

	\section{Context}
	Software systems complexity is increasing, and new technologies are introduced constantly. The software developers often have to work with new technologies which they are not familiar with, and as increasingly more comes out, the amount of information that they need to know will increase. Android, for example, was introduced 10 years ago in 2007 and nowadays there are more than three million applications available on the Google store. \\
	When Android started to become popular, developers had to learn this technology and to stay updated with each new version release. 
	They had to figure out how activities work in Android, and how to use several APIs to implement different tasks assigned to them. Where do they start from? Is there some tutorial on the web where they can learn how to use a specific API, or where they can find a piece of code that can be reused in their application? \\
	If we search on Google for an Android tutorial we will find 22 million documents\footnote{\url{https://www.google.ch/search?site=&source=hp&q=android+tutorial&oq=android+tutorial}}, even if we search for a specific field, as an Android Bluetooth tutorial\footnote{\url{https://www.google.ch/search?client=safari&rls=en&biw=1440&bih=839&q=android+bluetooth+example}}, we get 6 million documents which still a huge number.\\
	Stack Overflow is one of the most popular Q\&A websites for developers, where a million of questions are tagged as Android\footnote{\url{https://stackoverflow.com/questions/tagged/android}}. \\
	Github hosts more than 500 thousand Android repositories\footnote{\url{https://github.com/search?utf8=✓&q=android&type=}}, with millions of lines of code. Those numbers are gigantic, and it is obvious that finding the most suitable document is not obvious, since given the big number of available documents it is not feasible for the developer to check all of them and choose the right one.\\
	Therefore, the recommender systems for software engineering are an important support for programmers. They provide the must valuable documents in a given context.\\

	We have to introduce the proposed definition by the organizers of the ACM International Conference on Recommender Systems:\footnote{\url{https://recsys.acm.org/recsys09}} \\

	  \blockquote{\textit{``Recommendation systems are software applications that aim to support users in their decision-making while interacting with large information spaces. They recommend items of interest to users based on preferences they have expressed, either explicitly or implicitly. The ever-expanding volume and increasing complexity of information [...] has therefore made such systems essential tools for users in a variety of information seeking [...] activities. Recommendation systems help overcome the information overload problem by exposing users to the most interesting items, and by offering novelty, surprise, and relevance.''}}
	Now that we defined what is a recommender system, we can mention the recommender systems for software engineering (RSSE) definition given by \citet{RecommendationSystemsforSoftwareEngineering}:\\

	\textbf{An RSSE is a software application that provides information items estimated to be valuable for a software engineering task in a given context.}\\

	The RSSE definition highlights the importance of the \textbf{context} and the \textbf{valuable information}, therefore RSSEs are an important support for programmers to find the information they should know, and the RSSEs have to consider and evaluate alternative decisions. We believe that effort to comprehend a document must be a part of the document evaluation.\\


	Searching for documentation and tutorials is a crucial step in learning a new technology. The developers can find a bunch of online resources as blogs, forums, Q\&A websites, but the real challenge is to find the most suitable one for their needs.



	\citep{Singer-1997} reported in 1997 that the most frequent developer activity was code search, and \citep{Sadowski:2015} did a case study on how developers at Google search for code. They figured out that programmers are generally seeking answers to questions about how to use an API, what code does, why something is failing, or where the code is located. The interesting point in this study was the fact that most searches focus on code that is familiar or somewhat familiar to the developers.\\
	Therefore, we believe that recommender systems for software developers have to take into consideration the familiarity of a document when they suggest it to the developer. 


	Understanding a document is a cognitive process, and it depends on the human brain intelligence, but we all agree that if we are familiar with a subject, we will comprehend it with less effort. A computer engineer comprehends a document that explains how to implement a sorting algorithm, with much less effort compared to a document that explains a constitutional law, and the reason is not that the algorithm is not complex, but because a computer engineer is more familiar with sorting algorithms.\\
	
	In this study, we go further than this, and we try to evaluate more interesting situations, as for example:\\
	Given two documents that have the same subject, how can we decide which one is easier to comprehend?\\
	In order to answer this question we need to introduce two concepts:
	\begin{itemize}
	\item \textbf{familiarity}: how much are we familiar with the document content?
	\item \textbf{readability}: how difficult is it to read the document?
	\end{itemize}
	The comprehension effort can be derived from the document familiarity and readability.\\

	For example, if we have two documents where, in the first one we have a sorting algorithm implemented in Java, and in the second one a sorting algorithm implemented in Fortran, and the developer is more familiar with Java. And we want to use a tool that gives us a score that indicates which document requires less effort to be comprehended, where a higher score indicates a big effort.\\
	We expect that the first document must have a lower score since logically it requires less effort to be comprehended by developers who are more familiar with Java.
	What if both documents have a sorting algorithm implemented in the same programing language? which one will have a lower score? \\
	In this case, the document readability will have a big impact on the comprehension effort.\\
	Certainly, the developer will prefer to read the document with the best readability.\\
	
	\section{Objective and Results}
	Our main goal in this thesis is to assess documents by their comprehension effort, which can be used by RSSE to improve their suggestions.\\
	In order to calculate the comprehension effort, we need to find a way to calculate the familiarity, and we need to evaluate our approach to understand if it effectively works.\\

	 As a first step, we select a big set of Android documents, which represent the hypothetical developer knowledge, where each document contains code and natural language. Then we create a \textbf{ Language Model} that we train with these documents (training documents). In this way, we were able to simulate a programmer who is familiar with Android. Once we trained the language model, we evaluate the familiarity of a given set of documents (testing documents) that contains Android documents and other documents which are not related to Android.\\ 
	 The language model was able to evaluate the Android testing documents as thousand times more familiar than the rest of the testing documents.
	 In this experiment, the language model approach satisfied our expectation in capturing the document familiarity.\\

	 On the top of this experiment, we did a case study, where we give a set of tutorials to programmers, and after reading the tutorials, a set of documents is given to them, some documents are related to the tutorials and some are not. In this study the developers are just asked to assess the documents by the comprehension effort on a scale from 1 to 5 (1$-$low effort 5$-$high effort)
	 \newpage

	\section{Structure of the Thesis}
	This thesis consists of seven chapters: 
	\begin{enumerate}
	
		\item \textbf{Introduction}
		\item \textbf{State of the Art} describes the existing related work as code search engines and recommender systems.
		\item \textbf{Approach} describes our approach to calculate the comprehension effort.
		\item \textbf{Study design} in this chapter we discuss the research question, the data extraction process, and the analysis method, and the replication package 
		\item \textbf{Result} describes and discuss our experiment and case study results and their implication.
		\item \textbf{Threat to Validity} describes our assumptions, and the possible threats that could affect the results validity.
		\item \textbf{Conclusion} this chapter is a summarization of our work, where we present some ideas for a possible future work.
	\end{enumerate}
\chapter{State of the Art}
	In this chapter we talk about the existing related work to the comprehension effort, and we talk about code search engines and recommender systems.\\ 

	\citet{Kushwaha:2006:ICI:1163514.1163533} introduced a new equation :\\
	\begin{center}
	 \textbf{Difficulty in understanding the software}\\
	 \approx\\
	  \textbf{Difficulty in understanding the information}
	  \end{center}

	They claim that the required effort to understand a software depends on the difficulty in understanding the information, where the information is related to the number of operators and identifiers. They calculate the number of operators and identifiers per line of code and they multiply it by an associated weight of the identifier name ( 1 if the identifier name belongs to the problem domain, and 4 is the identifier name is selected arbitrarily). \\
	They performed an experiment on 60 students, where 5 sample programs where given. One set of programs had meaningful identifier names related to problem domain and the other used arbitrarily selected identifier name. They measured the required time to comprehend the program.\\ 
	The result showed that programs with arbitrarily selected identifier names required about 4 times the time to comprehend the programs compared to programs with meaningful identifier names.\\



	\newpage
	\citet{Scalabrino} tried to introduce a metric able to assess the understandability of a given snippet code. In their work they consider three types of metrics:
	\begin{enumerate}
		\item \textbf{Code-related metrics} are metrics related to the code, as cyclomatic complexity, LOC,  number of identifiers, line length and many other metrics.
		\item \textbf{Documentation-related metrics} captures the quality of the internal documentation of a snippet (comments readability, comments and identifiers consistency). 
		\item \textbf{Developer-related metrics} measures the programming experience of the developer in years, in any programming language.
	\end{enumerate}
	They analyze whether code-related, documentation-related, and developer- related metrics can be used to assess the understandability level of a snippet  code.
	46 developers participated to this study. They were asked to carefully read and to fully understand eight code snippets. Participants could, in any moment select the option \textit{I understood the snippet} or \textit{I cannot understand the snippet}, and the time was monitored.
	Once the participant chooses \textit{I understood the snippet} option , they ask question about the snippet code to verify the actual level of understanding.\\ 
	\textit{``After an extensive statistical analysis none of the considered metrics exhibit a significant correlation with the understandability of code snippets''.}\\
	They assumed that the code complexity has a big influence on the programmers' ability to understand the code, but they couldn't demonstrate it with a strong empirical evidence.\\
	They also mentioned that the code readability can have a direct impact on the understandability of the code.\\ As mentioned in the previous chapter, in this thesis the code readability is a part of our comprehension effort calculation.\\

	\citet{Buse2010} introduced a code readability metric, and they investigate its relation to software quality. A part of their work was to run an experiment which compares readability of the code to the cyclomatic complexity, and they were able to validate that the code readability is significantly independent from the traditional code complexity.\\
	In this experiment, 120 developers were asked to individually score a sequence of 100 code snippets, based on their personal estimation of readability. From the result they determined which code features are predictive of readability, and they construct a readability model. They also tested the model performance on ten different classifiers, and on average the model classified correctly between 75\% and 80\% of the snippets.\\
	They found that factors like \textit{average line length} and \textit{average number of identifiers per line} are very important to readability.\\
	In this thesis we use \citep{Buse2010} approach to calculate code readability.

	\section{Semantics Code Search}

	\citep{Reiss:2009:SCS:1555001.1555040} presented a tool that generates a specific functions or classes from the open source repositories, where these classes meet user's specifications. This tool uses the user's input, as keywords and other constraints, and suggests codes that meet the user's needs.\\
	The tool takes a set of candidate solutions, and it transforms it into a more appropriate set. Both static and dynamic specifications can be used.\\
	This tool main goal is to satisfy the user's constraint, but it doesn't take in consideration the user's experience or the comprehension effort.\\
	

	\citep{Thummalapenta:2007:PPA:1321631.1321663} presented a tool similar to \citep{Reiss:2009:SCS:1555001.1555040}, where they collect code from public sources and they they suggest it to the programmer based on their input query.\\ In the query the programmers have to specify the \textit{source object type} and the \textit{destination object type}.\\ This tool is too restrictive and doesn't take in consideration any readability or complexity metrics.\\

	\citep{journals/tse/McMillanGPFX12} created an application search system called \textit{Exemplar}, which reduce the mismatch between the high-level intent reflected in the descriptions of software and low-level implementation details.\\ Exemplar differs from the traditional search engine that matches the keywords, by matching keywords with the descriptions of the various API calls in help documents.\\
	Exemplar has three components of Ranking: 
	\begin{enumerate}
		\item \textbf{WOS} a component that computes a score based on word occurrences in project descriptions
		\item \textbf{RAS} a component that computes a score based on the relevant API calls
		\item \textbf{DCS} a score based on dataflow connections between calls
	\end{enumerate}
	In this paper \citet{journals/tse/McMillanGPFX12} said : \\
	\textit{``the results suggest that the performance of software search engines can be improved if those engines consider the API calls that the software uses.''}

	\section{Code Search Engines}
	\section{Libra}



\chapter{Approach}
	\section{Overview}
	\section{Language Model}
	\section{Stormed Island Parser}
	\section{Training the Language Model}
	\section{Accounting for Readability}

\chapter{Study Design}
	\section{Research Questions}
	\section{Data Collection and Analysis}	
	\section{Replication Package}

\chapter{Results}

\chapter{Threats to Validity}

\chapter{Conclusion}

\chapter[Short title]{A chapter title which will run over two lines --- it's for
  testing purpose}

\lipsum[1-2]

\section{The first section}
\lipsum[3-4]

 \section{The second, math section}

\textbf{Theorem 1 (Residue Theorem).}
Let $f$ be analytic in the region $G$ except for the isolated singularities $a_1,a_2,\ldots,a_m$. If $\gamma$ is a closed rectifiable curve in $G$ which does not pass through any of the points $a_k$ and if $\gamma\approx 0$ in $G$ then
\[
\frac{1}{2\pi i}\int_\gamma f = \sum_{k=1}^m n(\gamma;a_k) \text{Res}(f;a_k).
\]
\textbf{Theorem 2 (Maximum Modulus).}
\emph{Let $G$ be a bounded open set in $\mathbb{C}$ and suppose that $f$ is a continuous function on $G^-$ which is analytic in $G$. Then}
\[
\max\{|f(z)|:z\in G^-\}=\max \{|f(z)|:z\in \partial G \}.
\]

\section[third]{A very very long section, titled ``The third section'', with
  a rather  short text alternative (third)}
\lipsum \texttt{Some Test}
\lstset{language=algebra,linewidth=0.95\linewidth,breaklines=true,numbers=left,
basicstyle=\ttfamily,numberstyle=\tiny,escapeinside={//*}{\^^M},
mathescape=true}
\begin{lstlisting}
import IntSpec, ItemSpec;

sort cart; //*\label{sort}

constructors //*\label{begin-sig}
create() $\longrightarrow$ cart;
insert(cart, item) $\longrightarrow$ cart;
observers
amount(cart) $\longrightarrow$ int;
transformers
delete(cart, item) $\longrightarrow$ cart; //*\label{end-sig}

axioms //*\label{begin-axioms}
forall c: cart, i, j: item 

amount(create()) $=$ 0; //*\label{begin-amount}
amount(insert(c,i)) $=$ amount(c) $+$ price(i); //*\label{end-amount}
delete(create(),i) $=$ create(); //*\label{begin-delete}
delete(insert(c,i),j) $=$
if (i =$\:$= j) c
else insert(delete(c,j),i); //*\label{end-axioms}
end
\end{lstlisting}

As you can easily see from the above listing \citet{bbggs:iet07}
define something weird based on the BPEL specification
\citep{bpelspec}.
\cite{}
\nocite{*}

\appendix %optional, use only if you have an appendix

\chapter{Some retarded material}
\section{It's over\dots}
\lipsum 

\backmatter

\chapter{Glossary} %optional

%\bibliographystyle{alpha}
%\bibliographystyle{dcu}
\bibliographystyle{plainnat}
\bibliography{biblio}

%\cleardoublepage
%\theindex %optional, use only if you have an index, must use
	  %\makeindex in the preamble
\lipsum

\end{document}
